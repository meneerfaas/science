%--------------------------------------
% PACKAGES
%--------------------------------------

\usepackage{adjustbox}
\usepackage{caption}
\usepackage{cwpuzzle}
\usepackage{dashrule}
\usepackage[dutch]{babel}
\usepackage{enumitem}
\usepackage{etoolbox}
\usepackage{fontawesome5}
\usepackage{fontspec}
\usepackage{footnote}
\usepackage[edges]{forest}
\usepackage{graphicx}
\usepackage{listings}
\usepackage[os=win]{menukeys}
\usepackage{refcount}
\usepackage{siunitx}
\usepackage[skins]{tcolorbox}
\usepackage[countmax]{subfloat}
\usepackage{tabularray}
\usepackage{tikz}
\usepackage{unicode-math}
\usepackage{xcolor}
\usepackage{xr-hyper}
\usepackage{wrapfig}
\usepackage[hidelinks]{hyperref}
\usepackage[use-files]{xsim}
%\usepackage{lua-visual-debug}

\UseTblrLibrary{booktabs, counter, tikz}
\usetikzlibrary{%
    arrows,
    arrows.meta, 
    backgrounds,
    fit,
    matrix, 
    positioning, 
    shapes.geometric, 
    shapes.symbols, 
    tikzmark,
}
\usetikzmarklibrary{listings}

%--------------------------------------
% TITLE, AUTHOR, DATE
%--------------------------------------
\title{Programmeren met Python}
\author{Sander Faas}
\date{2026}

%--------------------------------------
% FLOATS on FLOAT PAGE to the TOP
%--------------------------------------

%https://texfaq.org/FAQ-vertposfp
\makeatletter
\setlength{\@fptop}{0pt}
\makeatother

%--------------------------------------
% BASIC PAGE LAYOUT
%--------------------------------------

% % Set page layout
% \setbinding{0.5cm}
% \setlrmarginsandblock{3cm}{2cm}{*}
% \setulmarginsandblock{2.5cm}{*}{1}
% % \setmarginnotes{17pt}{51pt}{\onelineskip}
% \setheadfoot{2\baselineskip}{2\baselineskip}
% \setheaderspaces{*}{*}{1.618}
% \checkandfixthelayout
% % Set paragraph indent and space between paragraphs
% \setlength{\parindent}{0em}
% \setlength{\parskip}{3pt}
% % Hyphenation
% \hyphenpenalty = 500 % lower means more hyphenation
% \raggedbottom % Prevents large spaces between paragraphs

% Set page layout
\settypeblocksize{22cm}{13cm}{*}
\setbinding{0.5cm}
\setlrmargins{3cm}{*}{*}
\setulmargins{4cm}{*}{*}
\setmarginnotes{17pt}{51pt}{\onelineskip}
\setheadfoot{2\baselineskip}{2\baselineskip}
\setheaderspaces{*}{*}{1.618}
\checkandfixthelayout
% Set paragraph indent and space between paragraphs
\setlength{\parindent}{0em}
\setlength{\parskip}{3pt}
% Hyphenation
\hyphenpenalty = 500 % lower means more hyphenation
\raggedbottom % Prevents large spaces between paragraphs

%--------------------------------------
% FONTS
%--------------------------------------

% Smallest acceptable font size
\renewcommand\RSsmallest{7pt}

% Set fonts (fontspec)
\defaultfontfeatures{Scale = MatchLowercase}
\setmainfont{Tex Gyre Pagella}
\setsansfont{Tex Gyre Heros}
\setmathfont{Tex Gyre Pagella Math}
\setmonofont{Source Code Pro}

% Font and color for title on cover page
\newfontfamily\titlefont{QTImpromptu}
\definecolor{covertitle}{HTML}{CDFFFF}
\definecolor{covertitleshadow}{HTML}{555555}

% Font for chapter headings
\newfontfamily\headingfont{Tex Gyre Heros}

% Font for programming code
\newfontfamily\codefont{Source Code Pro}
%\DeclareTextFontCommand{\code}{\codefont}

% Font for captions
\captiondelim{\par}
\captionnamefont{\small\sffamily\raggedleft}
\captiontitlefont{\small\sffamily\itshape\raggedleft}
\setlength{\belowcaptionskip}{3pt}

% Font for sidenotes
\renewcommand*{\sideparfont}{\sffamily\small}
\sideparmargin{outer}

% Visible space character (https://tex.stackexchange.com/questions/50804/explicit-space-character)
\newcommand\txtspc[1][.3em]{%
  \mbox{\kern.1em\vrule height.3ex}%
  \vbox{\hrule width#1}%
  \hbox{\vrule height.3ex\kern.1em}}

\newcommand\txtspcgray[1][.3em]{%
  \begingroup\color{black!40}%
  \mbox{\kern.1em\vrule height.3ex}%
  \vbox{\hrule width#1}%
  \hbox{\vrule height.3ex\kern.1em}%
  \endgroup%
}

%--------------------------------------
% CHAPTER STYLE
%--------------------------------------

\colorlet{chap_color}{blue!40!black}
\makechapterstyle{MyChapterStyle}{%
    % Set lengths for the chapter title
    \setlength{\beforechapskip}{10pt}
    \setlength{\midchapskip}{20pt}
    \setlength{\afterchapskip}{60pt}
    \renewcommand*{\chapnumfont}{\headingfont\bfseries\fontsize{5cm}{1cm}\color{chap_color}\flushright\selectfont}
    \renewcommand*{\chaptitlefont}{\headingfont\HUGE\bfseries\color{chap_color}\flushright}
    \renewcommand*{\printchaptername}{}
    \renewcommand*{\printchapternum}{\chapnumfont\thechapter}
}

%--------------------------------------
% SECTION STYLE
%--------------------------------------

\setsecheadstyle{\headingfont\Large\bfseries\color{chap_color}\raggedright}
\setsubsecheadstyle{\headingfont\large\bfseries\color{chap_color}\raggedright}

%--------------------------------------
% PAGE STYLE
%--------------------------------------

\makepagestyle{mypagestyle}
\nouppercaseheads
\makeevenhead{mypagestyle}{}{}{}
\makeoddhead{mypagestyle}{}{}{}
\makeevenfoot{mypagestyle}{\sffamily\bfseries\thepage}{}{\sffamily\leftmark}
\makeoddfoot{mypagestyle}{\sffamily\rightmark}{}{\sffamily\bfseries\thepage}

\copypagestyle{chapter}{mypagestyle}

%--------------------------------------
% FOOT NOTES
%--------------------------------------

\setlength{\footmarkwidth}{1em}
\setlength{\footmarksep}{0em}
\setlength{\footparindent}{0em}
\setfootins{1cm}{.5cm}

\footmarkstyle{\textsuperscript{#1}\hskip 0.5em}


%--------------------------------------
% TCOLORBOXES
%--------------------------------------

% Save and spew notes for tcolorboxes
% https://tex.stackexchange.com/questions/585246/footnotes-with-tcolorbox
\BeforeBeginEnvironment{tcolorbox}{\savenotes}
\AfterEndEnvironment{tcolorbox}{\spewnotes}
\BeforeBeginEnvironment{infobox}{\savenotes}
\AfterEndEnvironment{infobox}{\spewnotes}

\colorlet{ob_box_color}{blue!40!black}
\newtcolorbox{objectivesbox}{
    enhanced,
    drop fuzzy shadow,
    boxsep=2.5mm,
    title=In dit hoofdstuk leer je\dots,
    coltitle=ob_box_color,
    colbacktitle=ob_box_color!20!white,
    colback=white,
    colframe=ob_box_color,
    fonttitle=\sffamily\bfseries\large\color{ob_box_color},
    after skip=1cm,
    boxrule=1.5pt,
    overlay={
            \node[anchor=east] at (frame.north east) [
                rounded corners,
                minimum size=1.5cm,
                inner sep=0pt,
                xshift=-5mm,
                line width=1.5pt,
                fill=white,
                draw=ob_box_color] {\Huge\color{ob_box_color!80!black}\faEye};
        }
}

\newenvironment{objectiveslist}{
    \begin{itemize}[
            label=\tiny\color{ob_box_color!80!black}{\faEye},
            nosep,
            left=0pt,
            before=\sffamily
        ]
        }
        {
    \end{itemize}
}

\colorlet{info_box_color}{yellow!40!black}
\newtcolorbox{infobox}[1][Informatief]{
    enhanced,
    drop fuzzy shadow,
    boxsep=2.5mm,
    title=#1,
    coltitle=info_box_color,
    colbacktitle=info_box_color!20!white,
    colback=white,
    colframe=info_box_color,
    fonttitle=\sffamily\bfseries\large\color{info_box_color},
    before skip=1cm,
    after skip=1cm,
    boxrule=1.5pt,
    overlay={
            \node[anchor=east] at (frame.north east) [
                rounded corners,
                minimum size=1.5cm,
                inner sep=0pt,
                xshift=-5mm,
                line width=1.5pt,
                fill=white,
                draw=info_box_color] {\Huge\color{info_box_color!80!black}\faInfoCircle};
        }
}

\colorlet{tip_box_color}{blue!40!black}
\newtcolorbox{tipbox}[1][Tip]{
    enhanced,
    drop fuzzy shadow,
    boxsep=2.5mm,
    title=#1,
    coltitle=tip_box_color,
    colbacktitle=tip_box_color!20!white,
    colback=white,
    colframe=tip_box_color,
    fonttitle=\sffamily\bfseries\large\color{tip_box_color},
    before skip=1cm,
    after skip=1cm,
    boxrule=1.5pt,
    overlay={
            \node[anchor=east] at (frame.north east) [
                rounded corners,
                minimum size=1.5cm,
                inner sep=0pt,
                xshift=-5mm,
                line width=1.5pt,
                fill=white,
                draw=tip_box_color] {\Huge\color{tip_box_color!80!black}\faStyle{regular}\faLightbulb};
        }
}

% TColorbox for Mu Editor shell
\definecolor{mu_grey}{HTML}{EEEEEE}
\definecolor{mu_ivory}{HTML}{FEFEF7}
\newfontfamily\uifont{Segoe UI}
\NewTotalTColorBox{\shellbox}{ O{} m m }{%
    skin=bicolor,
    title={\hspace{1mm}Running: #2},
    frame style={
        top color=mu_grey,
        bottom color=mu_grey,
        draw=black!30,
    },
    interior style={
        draw=black!30,
    },
    segmentation style={
        draw=black!30,
    },
    colback=mu_ivory,
    colbacklower=mu_grey,
    coltitle=black,
    colupper=black,
    collower=black,
    fonttitle=\uifont\tiny,
    fontupper=\ttfamily,
    fontlower=\uifont\tiny,
    halign=flush left,
    halign lower=flush right,
    sharp corners,
    toptitle=1mm,
    bottomtitle=0mm,
    titlerule=0mm,
    left=1mm,
    leftrule=0mm,
    rightrule=0mm,
    bottomrule=0mm,
    right=2mm,
    bottom=0mm,
    boxsep=1mm,
    middle=0mm,
    #1,
    underlay={
    \node[anchor=north east] at ([xshift=-2mm]segmentation.north east) [
        text=black!50,
    ] (cog) {\faCog};
    \node[anchor=east] at (cog.west) [
        font=\uifont\tiny,
    ] {Python 3};
    \draw [black!20] ([yshift=-.5mm]cog.north east) -- ([yshift=.5mm]cog.south east);
    \draw [black!20] ([yshift=-.5mm]cog.north west) -- ([yshift=.5mm]cog.south west);
    },    
}{%
    #3
    \vspace{6pt}
    \tcblower
    \vspace{.5cm}
}

%--------------------------------------
% MENUKEYS
%--------------------------------------

%https://tex.stackexchange.com/questions/387952/create-windows-symbol-and-apple-logo-in-package-menukeys
\makeatletter
\tw@make@key@box{OS@mac}{\faApple}
\tw@make@key@box{OS@win}{\faWindows}
\tw@make@key@macro*{\OS}
\makeatother

\newmenumacro{\folder}[/]{pathswithfolder}
\newmenumacro{\file}[/]{paths}

%--------------------------------------
% FOREST
%--------------------------------------

%https://tex.stackexchange.com/a/405253/
\definecolor{folderbg}{RGB}{124,166,198}
\definecolor{folderborder}{RGB}{110,144,169}
\newlength\Size
\setlength\Size{4pt}

\tikzset{%
    folder/.pic={%
        \filldraw [draw=folderborder, top color=folderbg!50, bottom color=folderbg] (-1.05*\Size,0.2\Size+5pt) rectangle ++(.75*\Size,-0.2\Size);
        \filldraw [draw=folderborder, top color=folderbg!50, bottom color=folderbg] (-1.15*\Size,-\Size) rectangle (1.15*\Size,\Size);
    },
    file/.pic={%
        \filldraw [draw=folderborder, top color=folderbg!5, bottom color=folderbg!10] (-\Size,.4*\Size+5pt) coordinate (a) |- (\Size, -1.2*\Size) coordinate (b) -- ++(0,1.6*\Size) coordinate (c) -- ++(-5pt,5pt) coordinate (d) -- cycle (d) |- (c) ;
    },
}

\forestset{%
  declare autowrapped toks={pic me}{},
  declare boolean register={pic root},
  pic root=0,
  pic dir tree/.style={%
    for tree={%
      folder,
      font=\ttfamily,
      grow'=0,
    },
    before typesetting nodes={%
      for tree={%
        edge label+/.option={pic me},
      },
      if pic root={
        tikz+={
          \pic at ([xshift=\Size].west) {folder};
        },
        align={l}
      }{},
    },
  },
  pic me set/.code n args=2{%
    \forestset{%
      #1/.style={%
        inner xsep=2\Size,
        pic me={pic {#2}},
      }
    }
  },
  pic me set={directory}{folder},
  pic me set={file}{file},
}

%--------------------------------------
% LISTINGS
%--------------------------------------

\lstloadlanguages{[3]Python}

\definecolor{code_darkblue}{HTML}{0000A0}
\definecolor{code_maroon}{HTML}{800000}
\definecolor{code_sherpablue}{HTML}{005050}
\definecolor{code_grey}{HTML}{808080}
\definecolor{code_ivory}{HTML}{FEFEF7}

\def\keywordstyleifnotalreadystringstyle{%
    \extractcolorspec{.}\currentcolor
    \extractcolorspec{code_maroon}\stringcolor
    \ifx\currentcolor\stringcolor\else
        \bfseries\color{code_sherpablue}%
    \fi
}

\lstset{
    language=[3]Python,
    aboveskip=\baselineskip,
    backgroundcolor=\color{code_ivory},
    basicstyle=\ttfamily,
    breaklines=true,
    breakatwhitespace=true,
    captionpos=b,
    escapeinside={(*@}{@*)},
    frame=single,
    frameround=tttt,
    morekeywords=[2]{print},
    moredelim=[s][\color{black}]{\{}{\}},%necessary
    morestring=**[s]{f"}{"},%to style
    morestring=**[s]{f'}{'},%f-strings
    morestring=[d]{\\'},
    numberbychapter = true,
    numberstyle=\small\color{code_grey},
    showstringspaces=false,
}

\lstdefinestyle{shell}{%
    commentstyle=\color{black},
    keywordstyle=\color{black},
    keywordstyle=\color{black},
    numberstyle=\color{black},
    stringstyle=\color{black},
    numbers=none,
}

\lstdefinestyle{mu}{%
    commentstyle=\color{code_grey}\itshape,
    keywordstyle=\keywordstyleifnotalreadystringstyle,
    keywordstyle={[2]\color{code_darkblue}},
    numberstyle=\small\color{code_grey},
    stringstyle=\color{code_maroon},
    numbers=left,
}

% New listing environment to prevent breaking across pages
\lstnewenvironment{python}[1][]{%
    \noindent\minipage[t]{\linewidth}\lstset{#1}}{%
    \endminipage\hfill}

% New listing environment for breakable listings
\lstnewenvironment{python-breakable}[1][]{\lstset{#1}}{}

%--------------------------------------
% XSIM EXERCISES
%--------------------------------------

% Tags for icons
\DeclareExerciseTagging{icons}
\pgfkeyssetvalue{icons/laptop}{\faLaptopCode}
\pgfkeyssetvalue{icons/pen}{\faPencil*}

% Width of the exercise number node
\newlength{\exerciselabelwidth}
\setlength{\exerciselabelwidth}{.8cm}
\newlength{\solutionlabelwidth}
\setlength{\solutionlabelwidth}{.5cm}

% Exercise environment template 
\DeclareExerciseEnvironmentTemplate{exercise_tikz}
{%
    \edef\iconsize{1.4}% Used to calculate icon spacing
    \edef\iconsinrow{2}% Used to push icons to the next row if necessary
    \par\vspace{\IfInsideSolutionTF{0.1\baselineskip}{\baselineskip}}
    \begingroup\raggedright
    \trivlist
    \labelwidth=0pt
    \labelsep=0pt
    \item[{%
        \makebox[0pt][r]{%
            \smash{%
                \tikz[baseline=(label.base)]{%
                    \node[%
                        fill=chap_color,
                        text=white,
                        align=right,
                        font=\bfseries,
                        rounded corners,
                        inner sep=2mm,
                        text width=\IfInsideSolutionTF{\solutionlabelwidth}{\exerciselabelwidth},
                        outer sep=0pt,
                    ] (label) {\GetExerciseProperty{counter}};
                    \IfInsideSolutionF{
                        \edef\counter{0}
                        \edef\xshiftval{0}
                        \edef\yshiftval{0}
                        \ForEachExerciseTag{icons}{%
                            \node [%
                                text=chap_color,
                                below=1mm of label.south east,
                                anchor=north east,
                                xshift=\xshiftval em,
                                yshift=\yshiftval em,
                                outer sep=0pt,
                                inner sep=0pt
                            ] {\strut\pgfkeysvalueof{icons/#1}};
                            \pgfmathparse{\counter+1}
                            \xdef\counter{\pgfmathresult}
                            \pgfmathparse{int(mod(\counter,\iconsinrow))}
                            \xdef\xshiftval{-\iconsize*\pgfmathresult}
                            \pgfmathparse{int(\counter/\iconsinrow)}
                            \xdef\yshiftval{-\iconsize*\pgfmathresult}
                            }
                    }
                }\hspace{1em}
            }
        }%
    }]\relax
}{
    \endtrivlist
    \endgroup
}

% Exercise type opgave
\DeclareExerciseType{opgave}{
    exercise-env    = opgave,
    solution-env    = antwoord,
    exercise-name   = opgave,
    exercises-name  = opgaven,
    solution-name   = antwoord,
    solutions-name  = antwoorden,
    exercise-template  = exercise_tikz,
    solution-template  = exercise_tikz,
    exercise-heading    = \subsection*,
    solution-heading    = \subsection*,
    within = chapter,
    the-counter = \arabic{opgave}
}

% Store exercise files in separate folder
\xsimsetup{
    path={exercises},
    %antwoord/print = true % Just for testing purposes
}

%--------------------------------------
% QUESTION LISTS
%--------------------------------------

% To subdivide exercises into questions a, b, c, etc.

\newlist{qlist}{enumerate}{1}
\setlist[qlist]{%
    align=left,
    leftmargin=0.8cm,
    labelwidth=0.8cm,
    labelsep=0pt,
    itemsep=6pt,
    label=\sffamily\bfseries\large\textcolor{chap_color}{\alph*}
}
\newlist{alist}{enumerate}{1}
\setlist[alist]{%
    leftmargin=0.6cm,
    align=left,
    labelsep=*,
    itemsep=2pt,
    label=\sffamily\bfseries\textcolor{chap_color}{\alph*}
}

% To get the right width for tables in a qlist environment
% 0.35cm aquired by trial and error

\newlength{\tblrwidth}
\setlength{\tblrwidth}{\dimexpr\textwidth-0.8cm}

\newlist{tasklist}{itemize}{1}
\setlist[tasklist]{%
    align=left,
    leftmargin=0.5cm,
    label=\faCaretRight,
    labelsep=0pt,
    labelwidth=0.4cm,
    itemsep=2pt,
    topsep=3pt,
}

\newlist{hints}{itemize}{1}
\setlist[hints]{%
    align=left,
    leftmargin=0cm,
    labelwidth=0.8cm,
    labelsep=0pt,
    itemsep=2pt,
    label=\faLightbulb[regular],
    topsep=3pt,
    font=\large\color{chap_color},
}

%--------------------------------------
% QUESTION COUNTER
%--------------------------------------

% To enumerate questions in tblr environments

\preto\tblr{\setcounter{question}{0}}
\preto\longtblr{\setcounter{question}{0}}
\preto\opgave{\setcounter{question}{0}}
\newcounter{question}
\newcommand\qalph[1][\relax]{%
\ifx\relax#1\stepcounter{question}%
\else\setcounter{question}{#1}\fi%
\alph{question}}

%--------------------------------------
% TABULARRAY COMMANDS
%--------------------------------------

% Dotted clines (left, middle, right)
\NewTblrTableCommand\clineld[1]{\cline[.03em, dotted, rightpos=-1]{#1}}
\NewTblrTableCommand\clinemd[1]{\cline[.03em, dotted, leftpos=-1, rightpos=-1]{#1}}
\NewTblrTableCommand\clinerd[1]{\cline[.03em, dotted, leftpos=-1]{#1}}

% Solid clines (left, middle, right)
\NewTblrTableCommand\clinels[1]{\cline[.03em, rightpos=-1]{#1}}
\NewTblrTableCommand\clinems[1]{\cline[.03em, leftpos=-1, rightpos=-1]{#1}}
\NewTblrTableCommand\cliners[1]{\cline[.03em, leftpos=-1]{#1}}

% Top and bottom lines
\NewTblrTableCommand\topline{\hline[.08em]}
\NewTblrTableCommand\bottomline{\hline[.08em]}

% Empty theme for long tables
\NewTblrTheme{empty}{
  \DefTblrTemplate{foot}{empty}{}
  \SetTblrTemplate{foot}{empty}
  \DefTblrTemplate{head}{empty}{}
  \SetTblrTemplate{head}{empty}
}

%--------------------------------------
% DASHRULE COMMANDS
%--------------------------------------

% Draw a dotted hrule with same style as dotted clines

\newcommand{\hruled}[1][\linewidth]{\hdashrule{#1}{0.08ex}{.4pt 1pt}}

%--------------------------------------
% PUZZLE
%--------------------------------------

\PuzzleDefineColorCell{c}{black!15}

%--------------------------------------
% WRAPFIGURE INTEXTSEP SETTING
%--------------------------------------
%https://tex.stackexchange.com/questions/682412/adjust-intextsep-for-wrapfigure-only-continued/682415#682415

\makeatletter
\patchcmd\WF@putfigmaybe{\lower\intextsep}{}{}{\fail}%
\AddToHook{env/wrapfigure/begin}{\setlength{\intextsep}{0pt}}
\makeatother

%--------------------------------------
% FILLPIC FOR FULL PAGE PICTURE
%--------------------------------------
% https://tex.stackexchange.com/questions/516210/how-to-fill-page-with-a-picture

\newcommand\fillpic[1]{%
    \setbox0\hbox{\includegraphics*[keepaspectratio=true,]{#1}}%
    \begin{tikzpicture}[overlay,remember picture]
        \pgfmathsetmacro{\myscale}{max(\the\paperwidth/\the\wd0,\the\paperheight/\the\ht0)}%   
        \node at (current page.center){\includegraphics*[keepaspectratio=true, scale=\myscale]{#1}};
  \end{tikzpicture}}